\documentclass[11pt,]{article}
\usepackage[left=1in,top=1in,right=1in,bottom=1in]{geometry}
\newcommand*{\authorfont}{\fontfamily{phv}\selectfont}
\usepackage[]{mathpazo}


  \usepackage[T1]{fontenc}
  \usepackage[utf8]{inputenc}




\usepackage{abstract}
\renewcommand{\abstractname}{}    % clear the title
\renewcommand{\absnamepos}{empty} % originally center

\renewenvironment{abstract}
 {{%
    \setlength{\leftmargin}{0mm}
    \setlength{\rightmargin}{\leftmargin}%
  }%
  \relax}
 {\endlist}

\makeatletter
\def\@maketitle{%
  \newpage
%  \null
%  \vskip 2em%
%  \begin{center}%
  \let \footnote \thanks
    {\fontsize{18}{20}\selectfont\raggedright  \setlength{\parindent}{0pt} \@title \par}%
}
%\fi
\makeatother




\setcounter{secnumdepth}{0}

\usepackage{color}
\usepackage{fancyvrb}
\newcommand{\VerbBar}{|}
\newcommand{\VERB}{\Verb[commandchars=\\\{\}]}
\DefineVerbatimEnvironment{Highlighting}{Verbatim}{commandchars=\\\{\}}
% Add ',fontsize=\small' for more characters per line
\usepackage{framed}
\definecolor{shadecolor}{RGB}{248,248,248}
\newenvironment{Shaded}{\begin{snugshade}}{\end{snugshade}}
\newcommand{\KeywordTok}[1]{\textcolor[rgb]{0.13,0.29,0.53}{\textbf{#1}}}
\newcommand{\DataTypeTok}[1]{\textcolor[rgb]{0.13,0.29,0.53}{#1}}
\newcommand{\DecValTok}[1]{\textcolor[rgb]{0.00,0.00,0.81}{#1}}
\newcommand{\BaseNTok}[1]{\textcolor[rgb]{0.00,0.00,0.81}{#1}}
\newcommand{\FloatTok}[1]{\textcolor[rgb]{0.00,0.00,0.81}{#1}}
\newcommand{\ConstantTok}[1]{\textcolor[rgb]{0.00,0.00,0.00}{#1}}
\newcommand{\CharTok}[1]{\textcolor[rgb]{0.31,0.60,0.02}{#1}}
\newcommand{\SpecialCharTok}[1]{\textcolor[rgb]{0.00,0.00,0.00}{#1}}
\newcommand{\StringTok}[1]{\textcolor[rgb]{0.31,0.60,0.02}{#1}}
\newcommand{\VerbatimStringTok}[1]{\textcolor[rgb]{0.31,0.60,0.02}{#1}}
\newcommand{\SpecialStringTok}[1]{\textcolor[rgb]{0.31,0.60,0.02}{#1}}
\newcommand{\ImportTok}[1]{#1}
\newcommand{\CommentTok}[1]{\textcolor[rgb]{0.56,0.35,0.01}{\textit{#1}}}
\newcommand{\DocumentationTok}[1]{\textcolor[rgb]{0.56,0.35,0.01}{\textbf{\textit{#1}}}}
\newcommand{\AnnotationTok}[1]{\textcolor[rgb]{0.56,0.35,0.01}{\textbf{\textit{#1}}}}
\newcommand{\CommentVarTok}[1]{\textcolor[rgb]{0.56,0.35,0.01}{\textbf{\textit{#1}}}}
\newcommand{\OtherTok}[1]{\textcolor[rgb]{0.56,0.35,0.01}{#1}}
\newcommand{\FunctionTok}[1]{\textcolor[rgb]{0.00,0.00,0.00}{#1}}
\newcommand{\VariableTok}[1]{\textcolor[rgb]{0.00,0.00,0.00}{#1}}
\newcommand{\ControlFlowTok}[1]{\textcolor[rgb]{0.13,0.29,0.53}{\textbf{#1}}}
\newcommand{\OperatorTok}[1]{\textcolor[rgb]{0.81,0.36,0.00}{\textbf{#1}}}
\newcommand{\BuiltInTok}[1]{#1}
\newcommand{\ExtensionTok}[1]{#1}
\newcommand{\PreprocessorTok}[1]{\textcolor[rgb]{0.56,0.35,0.01}{\textit{#1}}}
\newcommand{\AttributeTok}[1]{\textcolor[rgb]{0.77,0.63,0.00}{#1}}
\newcommand{\RegionMarkerTok}[1]{#1}
\newcommand{\InformationTok}[1]{\textcolor[rgb]{0.56,0.35,0.01}{\textbf{\textit{#1}}}}
\newcommand{\WarningTok}[1]{\textcolor[rgb]{0.56,0.35,0.01}{\textbf{\textit{#1}}}}
\newcommand{\AlertTok}[1]{\textcolor[rgb]{0.94,0.16,0.16}{#1}}
\newcommand{\ErrorTok}[1]{\textcolor[rgb]{0.64,0.00,0.00}{\textbf{#1}}}
\newcommand{\NormalTok}[1]{#1}

\usepackage{graphicx,grffile}
\makeatletter
\def\maxwidth{\ifdim\Gin@nat@width>\linewidth\linewidth\else\Gin@nat@width\fi}
\def\maxheight{\ifdim\Gin@nat@height>\textheight\textheight\else\Gin@nat@height\fi}
\makeatother
% Scale images if necessary, so that they will not overflow the page
% margins by default, and it is still possible to overwrite the defaults
% using explicit options in \includegraphics[width, height, ...]{}
\setkeys{Gin}{width=\maxwidth,height=\maxheight,keepaspectratio}


\title{Unsupervised Algorithms in machine learning \thanks{Template taken from (\url{http://github.com/svmiller}).
\textbf{Corresponding author}:
\href{mailto:svmille@clemson.edu}{\nolinkurl{svmille@clemson.edu}}.}  }



\author{\Large Cesar Conejo Villalobos\vspace{0.05in} \newline\normalsize\emph{Data Scientist}  }


\date{}

\usepackage{titlesec}

\titleformat*{\section}{\normalsize\bfseries}
\titleformat*{\subsection}{\normalsize\itshape}
\titleformat*{\subsubsection}{\normalsize\itshape}
\titleformat*{\paragraph}{\normalsize\itshape}
\titleformat*{\subparagraph}{\normalsize\itshape}


\usepackage{natbib}
\bibliographystyle{apsr}
\usepackage[strings]{underscore} % protect underscores in most circumstances



\newtheorem{hypothesis}{Hypothesis}
\usepackage{setspace}


% set default figure placement to htbp
\makeatletter
\def\fps@figure{htbp}
\makeatother

\usepackage{hyperref}

% move the hyperref stuff down here, after header-includes, to allow for - \usepackage{hyperref}

\makeatletter
\@ifpackageloaded{hyperref}{}{%
\ifxetex
  \PassOptionsToPackage{hyphens}{url}\usepackage[setpagesize=false, % page size defined by xetex
              unicode=false, % unicode breaks when used with xetex
              xetex]{hyperref}
\else
  \PassOptionsToPackage{hyphens}{url}\usepackage[draft,unicode=true]{hyperref}
\fi
}

\@ifpackageloaded{color}{
    \PassOptionsToPackage{usenames,dvipsnames}{color}
}{%
    \usepackage[usenames,dvipsnames]{color}
}
\makeatother
\hypersetup{breaklinks=true,
            bookmarks=true,
            pdfauthor={Cesar Conejo Villalobos (Data Scientist)},
             pdfkeywords = {Unsupervised, algorithms, k-means, hierarchical classification, kaggle,
R, parallel},  
            pdftitle={Unsupervised Algorithms in machine learning},
            colorlinks=true,
            citecolor=blue,
            urlcolor=blue,
            linkcolor=magenta,
            pdfborder={0 0 0}}
\urlstyle{same}  % don't use monospace font for urls

% Add an option for endnotes. -----


% add tightlist ----------
\providecommand{\tightlist}{%
\setlength{\itemsep}{0pt}\setlength{\parskip}{0pt}}

% add some other packages ----------

% \usepackage{multicol}
% This should regulate where figures float
% See: https://tex.stackexchange.com/questions/2275/keeping-tables-figures-close-to-where-they-are-mentioned
\usepackage[section]{placeins}


\begin{document}
	
% \pagenumbering{arabic}% resets `page` counter to 1 
%
% \maketitle

{% \usefont{T1}{pnc}{m}{n}
\setlength{\parindent}{0pt}
\thispagestyle{plain}
{\fontsize{18}{20}\selectfont\raggedright 
\maketitle  % title \par  

}

{
   \vskip 13.5pt\relax \normalsize\fontsize{11}{12} 
\textbf{\authorfont Cesar Conejo Villalobos} \hskip 15pt \emph{\small Data Scientist}   

}

}








\begin{abstract}

    \hbox{\vrule height .2pt width 39.14pc}

    \vskip 8.5pt % \small 

\noindent This document provides some examples of unsupervised algorithms in
machine learning. In these techniques, we need to infer the properties
of the observations without the help of an output variable or
\emph{supervisor}. We review two methods: k-means and hierarchical
clustering. Then we use some data from Kaggle for applying these
techniques to produce a customer segmentation. The platform that we use
is R. Because of the number of observations, we are going to use a
parallel process for improving the execution times.


\vskip 8.5pt \noindent \emph{Keywords}: Unsupervised, algorithms, k-means, hierarchical classification, kaggle,
R, parallel \par

    \hbox{\vrule height .2pt width 39.14pc}



\end{abstract}


\vskip -8.5pt


 % removetitleabstract

\noindent  

\section{Introduction}\label{introduction}

In the book \emph{The Elements of Statistical Learning}
\citet{xie2013ddrk} explains that in the case of unsupervised learning,
data usually has a set of \(N\) observations \((x_{1}, ..., x_{N})\) of
a random vector \(X\) having joint density \(Pr(X)\). The goal is to
infer the properties of this probability density.

The techniques that statistics and machine learning offer us for
unsupervised learning are the following:

\begin{enumerate}
\def\labelenumi{\arabic{enumi})}
\tightlist
\item
  Principal components, multidimensional scaling.
\item
  Cluster analysis.
\item
  Mixture modeling.
\item
  Association rules.
\end{enumerate}

In this exercise, we are going to focus on cluster analysis. The basis
of our model will be the
\href{https://www.kaggle.com/arjunbhasin2013/ccdata}{Kaggle} Credit Card
dataset for Clustering. The data are an 8950 x 18 matrix. One variable
is categorical and represents the customer ID, the next seventeen are
real numbers each representing the behavior of credit cardholders. The
goal is to define a marketing strategy based on customer segmentation.

The first aspect we need to solve is to find the number of clusters that
we need. \citet{xie2013ddrk} says that we can have two scenarios:

\begin{enumerate}
\def\labelenumi{\arabic{enumi})}
\item
  For data segmentation, the number of clusters is defined as part of
  the problem, and it is base on the capacity and resources of the
  company. The goal is to find observations that belong to each proposed
  group.
\item
  Determine how the observations belong to natural distinct groupings.
  In this case, the number of clusters is unknown.
\end{enumerate}

For this exercise, we are going to use scenario 2 and trying to find the
number of clusters and the characteristics of each group using the
following techniques:

\begin{enumerate}
\def\labelenumi{\arabic{enumi})}
\tightlist
\item
  k-means
\item
  Hierarchical clustering.
\end{enumerate}

\section{Techniques}\label{techniques}

As mentioned before, the goal of unsupervised algorithms is to get the
classes as homogeneous as possible and such that they are sufficiently
separated. This goal can be specified numerically from the following
property:

Suppose that exist a partition \(P = (C_{1},..., C_{K})\) of \(\Omega\),
where \(g_{1},..., g_{K}\) are the cluster center of the classes:

\[g_{k} = \frac{1}{|C_{k}|} \sum_{i \in C_{k} }{x_{i}} \]

Also \(g\) is the global center
\(\frac{1}{N} \sum_{i = 1 }^{N} {x_{i}}\). We also define:

\begin{itemize}
\item
  Total point scatter:
  \(T = \frac{1}{N} \sum_{i = 1 }^{N} {|| x_{i} - g ||}^{2}\).
\item
  Within-cluster point scatter:
  \(W(C) = \frac{1}{N} \sum_{k = 1 }^{K} \sum_{i \in C_{k}} {|| x_{i} - g_{k} ||}^{2}\).
\item
  Between-cluster point scatter:
  \(B(C) = \sum_{k = 1}^{K} \frac{|C_{k}|}{N} {|| g_{k} - g ||}^{2}\)
\end{itemize}

In this case, the algorithm requires \(B(C)\) to be maximum and \(W(C)\)
to be minimum. Since the total point scatter \(T\) is fixed, then
maximizing \(B(C)\) automatically minimizes \(W(C)\). Therefore, the two
goals (homogeneity within classes and separation between classes) are
achieved at the same time by minimizing \(W(C)\).

Thus, the goal in the \emph{K-means} method is to find a partition \(C\)
of \(\Omega\). Also, we find some representatives of the classes, such
that \(W(C)\) is minimal. For determining how many clusters a dataset
has, we can use the elbow method.

Furthermore, k-means depend on the choice of the number of clusters. On
the other hand, hierarchical clustering methods do not expect such
designations. Instead, \citet{xie2013ddrk} claims that this method
demands the user to specify a measure of dissimilarity between
(disjoint) groups of observations, based on the pairwise dissimilarities
among the observations.

This method of classification uses a notion of proximity between groups
of elements to measure the separation between the classes sought. To do
this, the concept of aggregation is introduced, which is nothing more
than a dissimilarity between groups of individuals: be \(A\), \(B\)
\(\subseteq \Omega\) then the aggregation between \(A\) and \(B\) is
\(\delta(A,B)\). Then we have the following agglomerative clustering
methods:

\begin{itemize}
\item
  Single linkage:
  \(\delta_{SL}(A, B) = min\{d(x_{i}, d_{j})| x_{i} \in A, x_{j} \in B \}\)
\item
  Complete linkage:
  \(\delta_{CL}(A, B) = max\{d(x_{i}, d_{j})| x_{i} \in A, x_{j} \in B \}\)
\item
  Average linkage:
  \(\delta_{AL}(A, B) = \frac{1}{|A||B|} \sum_{x_{i} \in A, x_{j} \in B } d(x_{i}, d_{j})\)
\item
  Ward linkage:
  \(\delta_{Ward}(A, B) = \frac{|A||B|}{|A| + |B|} {|| g_{A} - g_{B} ||}^{2}\)
\end{itemize}

\section{Analysis}\label{analysis}

We are going to define the marketing strategy using k-means and
hierarchal clustering. But first, we will see the distribution of the
data.

\subsection{\texorpdfstring{\textbf{Exploratory
Analysis}}{Exploratory Analysis}}\label{exploratory-analysis}

We create the function \emph{\textbf{cc\_stats()}} for analyzing some of
the characteristics of the dataset such as:

\begin{itemize}
\item
  Number of complete observations.
\item
  Number of \texttt{NA} values.
\item
  Mean of complete observations.
\item
  Standard desviation of complete observations.
\item
  Number of outliers observations (\(Q3 + 1.5IQR\))
\item
  Minimun value of complete observations.
\item
  Maximun value of complete observations.
\item
  95 quantile
\item
  Upper limit for the value. (mean + 3 sd)
\end{itemize}

\begin{Shaded}
\begin{Highlighting}[]
\OperatorTok{---}
\CommentTok{# Basic statistics}
\CommentTok{# Input: x vector}
\CommentTok{# Output: Summary of statistics of the input}

\NormalTok{cc_stats <-}\StringTok{ }\ControlFlowTok{function}\NormalTok{(x)\{}
  
  \CommentTok{#NA Values}
\NormalTok{  nas =}\StringTok{ }\KeywordTok{sum}\NormalTok{(}\KeywordTok{is.na}\NormalTok{(x))}
  
  \CommentTok{# Vector with complete values}
\NormalTok{  a =}\StringTok{ }\NormalTok{x[}\OperatorTok{!}\KeywordTok{is.na}\NormalTok{(x)]}
  
  \CommentTok{# Properties}
  
\NormalTok{  m   =}\StringTok{ }\KeywordTok{mean}\NormalTok{(a)}
\NormalTok{  min =}\StringTok{ }\KeywordTok{min}\NormalTok{(a)}
\NormalTok{  max =}\StringTok{ }\KeywordTok{max}\NormalTok{(a)}
\NormalTok{  s   =}\StringTok{ }\KeywordTok{sd}\NormalTok{(a)}
  
  \CommentTok{# Stats}
\NormalTok{  stats <-}\StringTok{ }\KeywordTok{boxplot.stats}\NormalTok{(a)}
\NormalTok{  n     <-}\StringTok{ }\NormalTok{stats}\OperatorTok{$}\NormalTok{n}
\NormalTok{  out   <-}\StringTok{ }\KeywordTok{length}\NormalTok{(stats}\OperatorTok{$}\NormalTok{out)}
  
\NormalTok{  Q95 =}\StringTok{ }\KeywordTok{quantile}\NormalTok{(a, }\FloatTok{0.95}\NormalTok{)}
\NormalTok{  UL =}\StringTok{ }\NormalTok{m }\OperatorTok{+}\StringTok{ }\DecValTok{3}\OperatorTok{*}\NormalTok{s}
  
  \KeywordTok{return}\NormalTok{(}\KeywordTok{c}\NormalTok{(}\DataTypeTok{n     =}\NormalTok{ n,}
           \DataTypeTok{nas   =}\NormalTok{ nas,}
           \DataTypeTok{Mean  =}\NormalTok{ m,}
           \DataTypeTok{StDev =}\NormalTok{ s,}
           \DataTypeTok{Q_out =}\NormalTok{ out,   }
           \DataTypeTok{Min   =}\NormalTok{ min,}
           \DataTypeTok{Max   =}\NormalTok{ max,}
           \DataTypeTok{Q   =}\NormalTok{ Q95,}
           \DataTypeTok{Upper_Limit =}\NormalTok{ UL))}
\NormalTok{\}}
\end{Highlighting}
\end{Shaded}

Using the function \emph{\textbf{apply()}}, we see the statistical
characteristics for each of the variables:

\begin{Shaded}
\begin{Highlighting}[]
\CommentTok{# Vector with the name of the variable}
\NormalTok{vars <-}\StringTok{ }\KeywordTok{c}\NormalTok{(}\StringTok{"BALANCE"}\NormalTok{,}
          \StringTok{"BALANCE_FREQUENCY"}\NormalTok{,}
          \StringTok{"PURCHASES"}\NormalTok{,}
          \StringTok{"ONEOFF_PURCHASES"}\NormalTok{,}
          \StringTok{"INSTALLMENTS_PURCHASES"}\NormalTok{,          }
          \StringTok{"CASH_ADVANCE"}\NormalTok{,}
          \StringTok{"PURCHASES_FREQUENCY"}\NormalTok{,}
          \StringTok{"ONEOFF_PURCHASES_FREQUENCY"}\NormalTok{,}
          \StringTok{"PURCHASES_INSTALLMENTS_FREQUENCY"}\NormalTok{,}
          \StringTok{"CASH_ADVANCE_FREQUENCY"}\NormalTok{,}
          \StringTok{"CASH_ADVANCE_TRX"}\NormalTok{,}
          \StringTok{"PURCHASES_TRX"}\NormalTok{,}
          \StringTok{"CREDIT_LIMIT"}\NormalTok{,}
          \StringTok{"PAYMENTS"}\NormalTok{,}
          \StringTok{"MINIMUM_PAYMENTS"}\NormalTok{,}
          \StringTok{"PRC_FULL_PAYMENT"}\NormalTok{,}
          \StringTok{"TENURE"}\NormalTok{)}
\end{Highlighting}
\end{Shaded}

\begin{Shaded}
\begin{Highlighting}[]
\CommentTok{# Apply the function for each variable}
\NormalTok{describe_stats <-}\StringTok{ }\KeywordTok{t}\NormalTok{(}\KeywordTok{data.frame}\NormalTok{(}\KeywordTok{apply}\NormalTok{(cc_general[vars], }\DecValTok{2}\NormalTok{, cc_stats)))}
\NormalTok{describe_stats}
\end{Highlighting}
\end{Shaded}

\begin{verbatim}
##                                     n nas    Mean   StDev Q_out    Min     Max
## BALANCE                          8950   0 1564.47 2081.53   695  0.000 19043.1
## BALANCE_FREQUENCY                8950   0    0.88    0.24  1493  0.000     1.0
## PURCHASES                        8950   0 1003.20 2136.63   808  0.000 49039.6
## ONEOFF_PURCHASES                 8950   0  592.44 1659.89  1013  0.000 40761.2
## INSTALLMENTS_PURCHASES           8950   0  411.07  904.34   867  0.000 22500.0
## CASH_ADVANCE                     8950   0  978.87 2097.16  1030  0.000 47137.2
## PURCHASES_FREQUENCY              8950   0    0.49    0.40     0  0.000     1.0
## ONEOFF_PURCHASES_FREQUENCY       8950   0    0.20    0.30   782  0.000     1.0
## PURCHASES_INSTALLMENTS_FREQUENCY 8950   0    0.36    0.40     0  0.000     1.0
## CASH_ADVANCE_FREQUENCY           8950   0    0.14    0.20   525  0.000     1.5
## CASH_ADVANCE_TRX                 8950   0    3.25    6.82   804  0.000   123.0
## PURCHASES_TRX                    8950   0   14.71   24.86   766  0.000   358.0
## CREDIT_LIMIT                     8949   1 4494.45 3638.82   248 50.000 30000.0
## PAYMENTS                         8950   0 1733.14 2895.06   808  0.000 50721.5
## MINIMUM_PAYMENTS                 8637 313  864.21 2372.45   841  0.019 76406.2
## PRC_FULL_PAYMENT                 8950   0    0.15    0.29  1474  0.000     1.0
## TENURE                           8950   0   11.52    1.34  1366  6.000    12.0
##                                     Q.95% Upper_Limit
## BALANCE                           5909.11     7809.07
## BALANCE_FREQUENCY                    1.00        1.59
## PURCHASES                         3998.62     7413.11
## ONEOFF_PURCHASES                  2671.09     5572.10
## INSTALLMENTS_PURCHASES            1750.09     3124.08
## CASH_ADVANCE                      4647.17     7270.36
## PURCHASES_FREQUENCY                  1.00        1.69
## ONEOFF_PURCHASES_FREQUENCY           1.00        1.10
## PURCHASES_INSTALLMENTS_FREQUENCY     1.00        1.56
## CASH_ADVANCE_FREQUENCY               0.58        0.74
## CASH_ADVANCE_TRX                    15.00       23.72
## PURCHASES_TRX                       57.00       89.28
## CREDIT_LIMIT                     12000.00    15410.90
## PAYMENTS                          6082.09    10418.34
## MINIMUM_PAYMENTS                  2766.56     7981.55
## PRC_FULL_PAYMENT                     1.00        1.03
## TENURE                              12.00       15.53
\end{verbatim}

First of all, there is only a few values with \texttt{NA}. If we want to
see if they both happen at the same time, we can do:

\begin{Shaded}
\begin{Highlighting}[]
\KeywordTok{sum}\NormalTok{(}\KeywordTok{is.na}\NormalTok{(cc_general}\OperatorTok{$}\NormalTok{CREDIT_LIMIT) }\OperatorTok{&}\StringTok{ }\KeywordTok{is.na}\NormalTok{(cc_general}\OperatorTok{$}\NormalTok{MINIMUM_PAYMENTS))}
\end{Highlighting}
\end{Shaded}

\begin{verbatim}
## [1] 0
\end{verbatim}

As a result, the \texttt{NA} values do not occur in the same row. For
fixing these unknown values, we can follow three alternatives:

\begin{itemize}
\item
  Remove the cases.
\item
  Fill in the unknowns using some strategy.
\item
  Use tools that handle these types of values.
\end{itemize}

In this case, the unknown values only represent 3.51\%, so we decide
delete that observations.

\begin{Shaded}
\begin{Highlighting}[]
\NormalTok{cc_general <-}\StringTok{ }\NormalTok{cc_general[}\OperatorTok{-}\KeywordTok{which}\NormalTok{(}\KeywordTok{is.na}\NormalTok{(cc_general}\OperatorTok{$}\NormalTok{CREDIT_LIMIT) }
                              \OperatorTok{|}\StringTok{ }\KeywordTok{is.na}\NormalTok{(cc_general}\OperatorTok{$}\NormalTok{MINIMUM_PAYMENTS)),]}
\end{Highlighting}
\end{Shaded}

Other thing we see is that the variables are measure in diferent scales.
For example \emph{BALANCE FREQUENCY}, \emph{PURCHASES FREQUENCY},
\emph{ONE OFF PURCHASES FREQUENCY} and \emph{PURCHASES INSTALLMENTS
FREQUENCY} are measure with a score between 0 and 1. Other values are
measure in money units and others in number of transactions. Because
there are different units then we should scaling variables. We do that
with the function \emph{\textbf{normalize()}}:

\begin{Shaded}
\begin{Highlighting}[]
\OperatorTok{---}
\NormalTok{## Normalize}
\NormalTok{## Input: Numeric vector}
\NormalTok{## Output: Vector normalized.}

\NormalTok{normalize <-}\StringTok{ }\ControlFlowTok{function}\NormalTok{(x)\{}
  
\NormalTok{  min_x <-}\StringTok{ }\KeywordTok{min}\NormalTok{(x)}
\NormalTok{  max_x <-}\StringTok{ }\KeywordTok{max}\NormalTok{(x)}
  
  \KeywordTok{return}\NormalTok{((x }\OperatorTok{-}\StringTok{ }\NormalTok{min_x)}\OperatorTok{/}\NormalTok{(max_x }\OperatorTok{-}\StringTok{ }\NormalTok{min_x))}
  
\NormalTok{\}}
\end{Highlighting}
\end{Shaded}

Then, we apply the function to each variable

\begin{Shaded}
\begin{Highlighting}[]
\NormalTok{cc_general_norm     <-}\StringTok{ }\KeywordTok{data.frame}\NormalTok{(}\KeywordTok{apply}\NormalTok{(cc_general[vars], }\DecValTok{2}\NormalTok{, normalize))}
\end{Highlighting}
\end{Shaded}

Finally, we apply \emph{\textbf{cc\_stats()}} again for seeing the
changes in our data.

\begin{Shaded}
\begin{Highlighting}[]
\NormalTok{describe_stats_norm <-}\StringTok{ }\KeywordTok{t}\NormalTok{(}\KeywordTok{data.frame}\NormalTok{(}\KeywordTok{apply}\NormalTok{(cc_general_norm[vars], }\DecValTok{2}\NormalTok{, cc_stats)))}
\NormalTok{describe_stats_norm}
\end{Highlighting}
\end{Shaded}

\begin{verbatim}
##                                     n nas  Mean StDev Q_out Min Max Q.95%
## BALANCE                          8636   0 0.084 0.110   666   0   1 0.312
## BALANCE_FREQUENCY                8636   0 0.895 0.208  1511   0   1 1.000
## PURCHASES                        8636   0 0.021 0.044   767   0   1 0.083
## ONEOFF_PURCHASES                 8636   0 0.015 0.041   961   0   1 0.067
## INSTALLMENTS_PURCHASES           8636   0 0.019 0.041   811   0   1 0.080
## CASH_ADVANCE                     8636   0 0.021 0.045   976   0   1 0.100
## PURCHASES_FREQUENCY              8636   0 0.496 0.401     0   0   1 1.000
## ONEOFF_PURCHASES_FREQUENCY       8636   0 0.206 0.300   749   0   1 1.000
## PURCHASES_INSTALLMENTS_FREQUENCY 8636   0 0.369 0.398     0   0   1 1.000
## CASH_ADVANCE_FREQUENCY           8636   0 0.092 0.135   346   0   1 0.389
## CASH_ADVANCE_TRX                 8636   0 0.027 0.056   794   0   1 0.122
## PURCHASES_TRX                    8636   0 0.042 0.070   716   0   1 0.165
## CREDIT_LIMIT                     8636   0 0.149 0.122   243   0   1 0.399
## PAYMENTS                         8636   0 0.035 0.057   785   0   1 0.121
## MINIMUM_PAYMENTS                 8636   0 0.011 0.031   841   0   1 0.036
## PRC_FULL_PAYMENT                 8636   0 0.159 0.296  1343   0   1 1.000
## TENURE                           8636   0 0.922 0.218  1290   0   1 1.000
##                                  Upper_Limit
## BALANCE                                 0.41
## BALANCE_FREQUENCY                       1.52
## PURCHASES                               0.15
## ONEOFF_PURCHASES                        0.14
## INSTALLMENTS_PURCHASES                  0.14
## CASH_ADVANCE                            0.16
## PURCHASES_FREQUENCY                     1.70
## ONEOFF_PURCHASES_FREQUENCY              1.11
## PURCHASES_INSTALLMENTS_FREQUENCY        1.56
## CASH_ADVANCE_FREQUENCY                  0.50
## CASH_ADVANCE_TRX                        0.20
## PURCHASES_TRX                           0.25
## CREDIT_LIMIT                            0.52
## PAYMENTS                                0.21
## MINIMUM_PAYMENTS                        0.10
## PRC_FULL_PAYMENT                        1.05
## TENURE                                  1.58
\end{verbatim}

Now, all the variables are in a scale from 0 to 1. Also, there is no
change in the variance of the variables.

\subsection{\texorpdfstring{\textbf{K-means}}{K-means}}\label{k-means}

For applying the \emph{k-means}, we develop the following code. We set
the seed \(1234\) for reproducibility purposes.

\begin{Shaded}
\begin{Highlighting}[]
\KeywordTok{set.seed}\NormalTok{(}\DecValTok{1234}\NormalTok{)}
\end{Highlighting}
\end{Shaded}

First, we need decide the number of cluster. We apply K-means clustering
to the data using the following techniques:

\begin{itemize}
\item
  Hartigan-Wong
\item
  MacQueen
\item
  Lloyd
\item
  Forgy
\end{itemize}

\begin{Shaded}
\begin{Highlighting}[]
\KeywordTok{library}\NormalTok{(snow)}

\CommentTok{# How many k?}

\NormalTok{cl <-}\StringTok{ }\KeywordTok{makeCluster}\NormalTok{(}\DecValTok{4}\NormalTok{, }\DataTypeTok{type=}\StringTok{"SOCK"}\NormalTok{)}

\NormalTok{ignore <-}\StringTok{ }\KeywordTok{clusterEvalQ}\NormalTok{(cl, \{}\KeywordTok{library}\NormalTok{(MASS); }\OtherTok{NULL}\NormalTok{\}) }

\CommentTok{#Hartigan-Wong}
\NormalTok{results_HW <-}\StringTok{ }\KeywordTok{lapply}\NormalTok{(}\KeywordTok{seq}\NormalTok{(}\DecValTok{1}\NormalTok{,}\DecValTok{20}\NormalTok{), }\ControlFlowTok{function}\NormalTok{(x) }\KeywordTok{kmeans}\NormalTok{(cc_general_norm,}
                                                   \DataTypeTok{centers =}\NormalTok{ x,}
                                                   \DataTypeTok{algorithm =} \StringTok{"Hartigan-Wong"}\NormalTok{,}
                                                   \DataTypeTok{nstart =} \DecValTok{20}\NormalTok{))}

\NormalTok{variance_HW <-}\StringTok{ }\KeywordTok{sapply}\NormalTok{(results_HW, }\ControlFlowTok{function}\NormalTok{(results_HW) results_HW}\OperatorTok{$}\NormalTok{tot.withinss)}

\CommentTok{# MacQueen}
\NormalTok{results_MQ <-}\StringTok{ }\KeywordTok{lapply}\NormalTok{(}\KeywordTok{seq}\NormalTok{(}\DecValTok{1}\NormalTok{,}\DecValTok{20}\NormalTok{), }\ControlFlowTok{function}\NormalTok{(x) }\KeywordTok{kmeans}\NormalTok{(cc_general_norm,}
                                                   \DataTypeTok{centers =}\NormalTok{ x,}
                                                   \DataTypeTok{algorithm =} \StringTok{"MacQueen"}\NormalTok{,}
                                                   \DataTypeTok{nstart =} \DecValTok{20}\NormalTok{))}

\NormalTok{variance_MQ <-}\StringTok{ }\KeywordTok{sapply}\NormalTok{(results_MQ, }\ControlFlowTok{function}\NormalTok{(results_MQ) results_MQ}\OperatorTok{$}\NormalTok{tot.withinss)}

\CommentTok{# Lloyd}
\NormalTok{results_Ll <-}\StringTok{ }\KeywordTok{lapply}\NormalTok{(}\KeywordTok{seq}\NormalTok{(}\DecValTok{1}\NormalTok{,}\DecValTok{20}\NormalTok{), }\ControlFlowTok{function}\NormalTok{(x) }\KeywordTok{kmeans}\NormalTok{(cc_general_norm,}
                                                   \DataTypeTok{centers =}\NormalTok{ x,}
                                                   \DataTypeTok{algorithm =} \StringTok{"Lloyd"}\NormalTok{,}
                                                   \DataTypeTok{nstart =} \DecValTok{20}\NormalTok{))}

\NormalTok{variance_Ll <-}\StringTok{ }\KeywordTok{sapply}\NormalTok{(results_Ll, }\ControlFlowTok{function}\NormalTok{(results_Ll) results_Ll}\OperatorTok{$}\NormalTok{tot.withinss)}

\CommentTok{# Forgy}
\NormalTok{results_Fg <-}\StringTok{ }\KeywordTok{lapply}\NormalTok{(}\KeywordTok{seq}\NormalTok{(}\DecValTok{1}\NormalTok{,}\DecValTok{20}\NormalTok{), }\ControlFlowTok{function}\NormalTok{(x) }\KeywordTok{kmeans}\NormalTok{(cc_general_norm,}
                                                   \DataTypeTok{centers =}\NormalTok{ x,}
                                                   \DataTypeTok{algorithm =} \StringTok{"Forgy"}\NormalTok{,}
                                                   \DataTypeTok{nstart =} \DecValTok{20}\NormalTok{))}

\NormalTok{variance_Fg <-}\StringTok{ }\KeywordTok{sapply}\NormalTok{(results_Fg, }\ControlFlowTok{function}\NormalTok{(results_Fg) results_Fg}\OperatorTok{$}\NormalTok{tot.withinss)}

\KeywordTok{stopCluster}\NormalTok{(cl)}
\end{Highlighting}
\end{Shaded}

Then, we can observe the total within-cluster sum of squares for K-means
clustering for some cluster from 1 to 20.

\begin{Shaded}
\begin{Highlighting}[]
\KeywordTok{plot}\NormalTok{(variance_HW, }
     \DataTypeTok{col =} \StringTok{"red"}\NormalTok{,}
     \DataTypeTok{type =} \StringTok{"b"}\NormalTok{, }
     \DataTypeTok{xlab =} \StringTok{"Number of cluster k"}\NormalTok{,}
     \DataTypeTok{ylab =} \StringTok{"Sum of squares"}\NormalTok{, }
     \DataTypeTok{main =} \StringTok{"Elbow Method: No. of clusters by algorithm"}\NormalTok{)}
\KeywordTok{points}\NormalTok{(variance_MQ, }\DataTypeTok{col =} \StringTok{"blue"}\NormalTok{, }\DataTypeTok{type =} \StringTok{"b"}\NormalTok{)}
\KeywordTok{points}\NormalTok{(variance_Ll, }\DataTypeTok{col =} \StringTok{"green"}\NormalTok{, }\DataTypeTok{type =} \StringTok{"b"}\NormalTok{)}
\KeywordTok{points}\NormalTok{(variance_Fg, }\DataTypeTok{col =} \StringTok{"magenta"}\NormalTok{, }\DataTypeTok{type =} \StringTok{"b"}\NormalTok{)}
\KeywordTok{legend}\NormalTok{(}\StringTok{"topright"}\NormalTok{,}
       \DataTypeTok{legend =} \KeywordTok{c}\NormalTok{(}\StringTok{"Hartigan"}\NormalTok{,}\StringTok{"MacQueen"}\NormalTok{,}\StringTok{"Lloyd"}\NormalTok{,}\StringTok{"Forgy"}\NormalTok{), }
       \DataTypeTok{col =} \KeywordTok{c}\NormalTok{(}\StringTok{"red"}\NormalTok{, }\StringTok{"blue"}\NormalTok{, }\StringTok{"green"}\NormalTok{, }\StringTok{"magenta"}\NormalTok{), }
       \DataTypeTok{lty =} \DecValTok{1}\NormalTok{, }
       \DataTypeTok{lwd =} \DecValTok{1}\NormalTok{)}
\end{Highlighting}
\end{Shaded}

We can see that the kink occur at \(k = 4\), so this is the number of
cluster that we propose for the marketing strategy.
\includegraphics{C:/Users/CCONE/Documents/Cesar_Conejo/03-Data_Science/00-DS_Projects/Versionamiento_R/clustering/output/graphs/p1_elbow_method_norm.png}

Now, the question that we need to answer is which method for k-means to
use. Therefore, we code the following lines that shows the results of
the clustering in 4 groups.

\begin{Shaded}
\begin{Highlighting}[]
\CommentTok{# Which method:}

\CommentTok{# Hartigan-Wong}
\NormalTok{results_HW <-}\StringTok{ }\KeywordTok{lapply}\NormalTok{(}\KeywordTok{seq}\NormalTok{(}\DecValTok{1}\NormalTok{,}\DecValTok{20}\NormalTok{), }\ControlFlowTok{function}\NormalTok{(x) }\KeywordTok{kmeans}\NormalTok{(cc_general_norm, }
                                                   \DataTypeTok{centers =} \DecValTok{4}\NormalTok{,}
                                                   \DataTypeTok{nstart =} \DecValTok{20}\NormalTok{,}
                                                   \DataTypeTok{algorithm =} \StringTok{"Hartigan-Wong"}\NormalTok{))}
\NormalTok{betweenss_HW <-}\StringTok{ }\KeywordTok{sapply}\NormalTok{(results_HW, }\ControlFlowTok{function}\NormalTok{(results_HW) results_HW}\OperatorTok{$}\NormalTok{betweenss)}

\CommentTok{# MacQueen}
\NormalTok{results_MQ <-}\StringTok{ }\KeywordTok{lapply}\NormalTok{(}\KeywordTok{seq}\NormalTok{(}\DecValTok{1}\NormalTok{,}\DecValTok{20}\NormalTok{), }\ControlFlowTok{function}\NormalTok{(x) }\KeywordTok{kmeans}\NormalTok{(cc_general_norm, }
                                                   \DataTypeTok{centers =} \DecValTok{4}\NormalTok{,}
                                                   \DataTypeTok{nstart =} \DecValTok{20}\NormalTok{,}
                                                   \DataTypeTok{algorithm =} \StringTok{"MacQueen"}\NormalTok{))}
\NormalTok{betweenss_MQ <-}\StringTok{ }\KeywordTok{sapply}\NormalTok{(results_MQ, }\ControlFlowTok{function}\NormalTok{(results_MQ) results_MQ}\OperatorTok{$}\NormalTok{betweenss)}


\CommentTok{# Lloyd}
\NormalTok{results_Ll <-}\StringTok{ }\KeywordTok{lapply}\NormalTok{(}\KeywordTok{seq}\NormalTok{(}\DecValTok{1}\NormalTok{,}\DecValTok{20}\NormalTok{), }\ControlFlowTok{function}\NormalTok{(x) }\KeywordTok{kmeans}\NormalTok{(cc_general_norm, }
                                                   \DataTypeTok{centers =} \DecValTok{4}\NormalTok{,}
                                                   \DataTypeTok{nstart =} \DecValTok{20}\NormalTok{,}
                                                   \DataTypeTok{algorithm =} \StringTok{"Lloyd"}\NormalTok{))}
\NormalTok{betweenss_Ll <-}\StringTok{ }\KeywordTok{sapply}\NormalTok{(results_Ll, }\ControlFlowTok{function}\NormalTok{(results_Ll) results_Ll}\OperatorTok{$}\NormalTok{betweenss)}


\CommentTok{# Forgy}
\NormalTok{results_Fg <-}\StringTok{ }\KeywordTok{lapply}\NormalTok{(}\KeywordTok{seq}\NormalTok{(}\DecValTok{1}\NormalTok{,}\DecValTok{20}\NormalTok{), }\ControlFlowTok{function}\NormalTok{(x) }\KeywordTok{kmeans}\NormalTok{(cc_general_norm, }
                                                   \DataTypeTok{centers =} \DecValTok{4}\NormalTok{,}
                                                   \DataTypeTok{nstart =} \DecValTok{20}\NormalTok{,}
                                                   \DataTypeTok{algorithm =} \StringTok{"Forgy"}\NormalTok{))}
\NormalTok{betweenss_Fg <-}\StringTok{ }\KeywordTok{sapply}\NormalTok{(results_Fg, }\ControlFlowTok{function}\NormalTok{(results_Fg) results_Fg}\OperatorTok{$}\NormalTok{betweenss)}



\NormalTok{AVG_Between_Class <-}\StringTok{ }\KeywordTok{data.frame}\NormalTok{(}\DataTypeTok{Method =} \KeywordTok{c}\NormalTok{(}\StringTok{"Hartigan-Wong"}\NormalTok{, }\StringTok{"MacQueen"}\NormalTok{, }
                                           \StringTok{"Lloyd"}\NormalTok{, }\StringTok{"Forgy"}\NormalTok{),}
                     \DataTypeTok{AVG_Between_Class =} \KeywordTok{c}\NormalTok{(}\KeywordTok{mean}\NormalTok{(betweenss_HW), }\KeywordTok{mean}\NormalTok{(betweenss_MQ),}
                                           \KeywordTok{mean}\NormalTok{(betweenss_Ll),}\KeywordTok{mean}\NormalTok{(betweenss_Fg))}
\NormalTok{)}
\end{Highlighting}
\end{Shaded}

\begin{Shaded}
\begin{Highlighting}[]

\NormalTok{Method        | Avg between}
\NormalTok{------------- | -------------}
\NormalTok{Hartigan-Wong | 3222.460}
\NormalTok{MacQueen      | 3221.499}
\NormalTok{Lloyd         | 3220.927}
\NormalTok{Forgy         | 3221.112}
\end{Highlighting}
\end{Shaded}

Because Lloyd reachs the minimun average between-cluster point scatter,
we choose that method. As a result, we code the method in the following
way:

\begin{Shaded}
\begin{Highlighting}[]
\NormalTok{Cluster_FG <-}\StringTok{ }\KeywordTok{kmeans}\NormalTok{(}\DataTypeTok{x =}\NormalTok{ cc_general_norm,}
                     \DataTypeTok{centers =}  \DecValTok{4}\NormalTok{,}
                     \DataTypeTok{iter.max =} \DecValTok{100}\NormalTok{,}
                     \DataTypeTok{nstart =} \DecValTok{20}\NormalTok{,}
                     \DataTypeTok{algorithm =} \StringTok{"Lloyd"}\NormalTok{)}
\end{Highlighting}
\end{Shaded}

Finally, we can see some properties of the cluster. For example, the
number of observations by cluster is:

\begin{Shaded}
\begin{Highlighting}[]
\NormalTok{Cluster_FG}\OperatorTok{$}\NormalTok{size}
\end{Highlighting}
\end{Shaded}

\begin{verbatim}
## [1]  896 1249 2146 4345
\end{verbatim}

In this case, we can see that the classes are well balanced. Also, we
can see the center of each cluster and realize some interpretations.

\begin{Shaded}
\begin{Highlighting}[]
\KeywordTok{barplot}\NormalTok{(}\KeywordTok{t}\NormalTok{(Cluster_FG}\OperatorTok{$}\NormalTok{centers),}
        \DataTypeTok{main =} \StringTok{"Centers by cluster"}\NormalTok{,}
        \DataTypeTok{xlab =} \StringTok{"Cluster"}\NormalTok{,}
        \DataTypeTok{beside =} \OtherTok{TRUE}\NormalTok{, }
        \DataTypeTok{col =} \KeywordTok{rainbow}\NormalTok{(}\DecValTok{17}\NormalTok{)}
\NormalTok{        )}
\end{Highlighting}
\end{Shaded}

\includegraphics{figs/unnamed-chunk-15.pdf}

\subsection{Hierachical cluster}\label{hierachical-cluster}

\begin{Shaded}
\begin{Highlighting}[]
\NormalTok{model_Ward <-}\StringTok{ }\KeywordTok{hclust}\NormalTok{(}\KeywordTok{dist}\NormalTok{(cc_general_norm), }\DataTypeTok{method =} \StringTok{"ward.D2"}\NormalTok{)}

\KeywordTok{plot}\NormalTok{(model_Ward, }\DataTypeTok{labels =} \OtherTok{FALSE}\NormalTok{)}
\KeywordTok{rect.hclust}\NormalTok{(model_Ward, }\DataTypeTok{k =} \DecValTok{4}\NormalTok{, }\DataTypeTok{border =} \StringTok{"blue"}\NormalTok{)}
\end{Highlighting}
\end{Shaded}

\includegraphics{figs/unnamed-chunk-16.pdf}

\begin{Shaded}
\begin{Highlighting}[]
\NormalTok{cluster <-}\StringTok{ }\KeywordTok{cutree}\NormalTok{(model_Ward, }\DataTypeTok{k =} \DecValTok{4}\NormalTok{)}

\NormalTok{cc_general_norm_cluster <-}\KeywordTok{cbind}\NormalTok{(cc_general_norm, cluster)}

\KeywordTok{library}\NormalTok{(rattle)}

\NormalTok{cc_general_center <-}\StringTok{ }\KeywordTok{centers.hclust}\NormalTok{(cc_general_norm,}
\NormalTok{                                    model_Ward,}
                                    \DataTypeTok{nclust =} \DecValTok{4}\NormalTok{,}
                                    \DataTypeTok{use.median =} \OtherTok{FALSE}\NormalTok{)}
\end{Highlighting}
\end{Shaded}

\begin{Shaded}
\begin{Highlighting}[]
\KeywordTok{barplot}\NormalTok{(}\KeywordTok{t}\NormalTok{(cc_general_center),}
        \DataTypeTok{beside =} \OtherTok{TRUE}\NormalTok{,}
        \DataTypeTok{main =} \StringTok{"Cluster Interpretation"}\NormalTok{,}
        \DataTypeTok{col =} \KeywordTok{rainbow}\NormalTok{(}\DecValTok{17}\NormalTok{)}
\NormalTok{        )}
\end{Highlighting}
\end{Shaded}

\includegraphics{figs/unnamed-chunk-18.pdf}

\begin{Shaded}
\begin{Highlighting}[]
\NormalTok{## radar graph}
\NormalTok{center  <-}\StringTok{ }\KeywordTok{as.data.frame}\NormalTok{(cc_general_center)}
\NormalTok{maximos <-}\StringTok{ }\KeywordTok{apply}\NormalTok{(center,}\DecValTok{2}\NormalTok{,max)}
\NormalTok{minimos <-}\StringTok{ }\KeywordTok{apply}\NormalTok{(center,}\DecValTok{2}\NormalTok{,min)}
\NormalTok{center  <-}\StringTok{ }\KeywordTok{rbind}\NormalTok{(minimos,center)}
\NormalTok{center  <-}\StringTok{ }\KeywordTok{rbind}\NormalTok{(maximos,center)}
\end{Highlighting}
\end{Shaded}

\begin{Shaded}
\begin{Highlighting}[]
\KeywordTok{library}\NormalTok{(fmsb)}
\KeywordTok{radarchart}\NormalTok{(center,}
           \DataTypeTok{maxmin =} \OtherTok{TRUE}\NormalTok{,}
           \DataTypeTok{axistype =} \DecValTok{4}\NormalTok{,}
           \DataTypeTok{axislabcol =} \StringTok{"slategray4"}\NormalTok{,}
           \DataTypeTok{centerzero =} \OtherTok{FALSE}\NormalTok{,}
           \DataTypeTok{seg =} \DecValTok{8}\NormalTok{,}
           \DataTypeTok{cglcol =} \StringTok{"gray67"}\NormalTok{,}
           \DataTypeTok{pcol=}\KeywordTok{c}\NormalTok{(}\StringTok{"green"}\NormalTok{,}\StringTok{"blue"}\NormalTok{,}\StringTok{"red"}\NormalTok{, }\StringTok{"purple"}\NormalTok{, }\StringTok{"black"}\NormalTok{, }\StringTok{"brown"}\NormalTok{),}
           \DataTypeTok{plty =} \DecValTok{1}\NormalTok{,}
           \DataTypeTok{plwd =} \DecValTok{5}\NormalTok{,}
           \DataTypeTok{title =} \StringTok{"Comparación de clústeres")}


\StringTok{legenda <-legend(1.5,1, legend=c("}\NormalTok{Cluster }\DecValTok{1}\StringTok{","}\NormalTok{Cluster }\DecValTok{2}\StringTok{","}\NormalTok{Cluster }\DecValTok{3}\StringTok{", "}\NormalTok{Cluster }\DecValTok{4}\StringTok{"),}
\StringTok{                 seg.len=-1.4,}
\StringTok{                 title="}\NormalTok{Clústeres",}
                 \DataTypeTok{pch=}\DecValTok{21}\NormalTok{, }
                 \DataTypeTok{bty=}\StringTok{"n"}\NormalTok{ ,}\DataTypeTok{lwd=}\DecValTok{3}\NormalTok{, }\DataTypeTok{y.intersp=}\DecValTok{1}\NormalTok{, }\DataTypeTok{horiz=}\OtherTok{FALSE}\NormalTok{,}
                 \DataTypeTok{col=}\KeywordTok{c}\NormalTok{(}\StringTok{"green"}\NormalTok{,}\StringTok{"blue"}\NormalTok{,}\StringTok{"red"}\NormalTok{, }\StringTok{"purple"}\NormalTok{, }\StringTok{"black"}\NormalTok{, }\StringTok{"brown"}\NormalTok{)}
\NormalTok{                 )}
\end{Highlighting}
\end{Shaded}

\includegraphics{figs/unnamed-chunk-20.pdf}

\section{Conclusion}\label{conclusion}





\newpage
\singlespacing 
\bibliography{master.bib}

\end{document}
